%%%%%%%%%%%%%%%%%%%%%%%%%%%%%%%%%%%%%%%%%%%%%%%%%%%%%%%%%%%%%%%%%%%%%%%%%%%%%%%%
%2345678901234567890123456789012345678901234567890123456789012345678901234567890
%        1         2         3         4         5         6         7         8

\documentclass[letterpaper, 10 pt, conference]{ieeeconf}  % Comment this line out
                                                          % if you need a4paper
%\documentclass[a4paper, 10pt, conference]{ieeeconf}      % Use this line for a4
                                                          % paper

\IEEEoverridecommandlockouts                              % This command is only
                                                          % needed if you want to
                                                          % use the \thanks command
\overrideIEEEmargins
% See the \addtolength command later in the file to balance the column lengths
% on the last page of the document

\usepackage[utf8]{inputenc}
\usepackage[T1]{fontenc}
% ref packages
%\usepackage{nameref}
%\usepackage{varioref}
% \usepackage{hyperref}
\usepackage[hyphens,spaces,obeyspaces]{url} % Fixes long URLs
\usepackage{hyperref} % Let URLs be clickable and lead to the website
\usepackage{hhline} % <-- Add this line to use hhline package
\usepackage{booktabs}
\usepackage{colortbl}
\usepackage{xcolor}
% \usepackage{enumitem} % Let enumerations start with the alphabet
\usepackage{graphicx}
\usepackage{floatrow}
\usepackage{float}
\usepackage{aliascnt}
\usepackage{multirow} % Tables
\usepackage{array}
\newcolumntype{P}[1]{>{\centering\arraybackslash}p{#1}}
\usepackage{authblk} % For the author section 
\usepackage{caption} % advanced captions
\usepackage{amsmath}
%Mathebib
\RequirePackage{amsmath}
\RequirePackage{amssymb}
\usepackage[
style=ieee, % Zitierstil
isbn=false,
doi=false,
url=false,
pagetracker=true,
%autocite=inline,  % regelt Aussehen für \autocite (inline=\parancite)
block=space,               
backref=false,
backrefstyle=three+,
date=year,
backend=biber
]{biblatex}
\addbibresource{literature.bib}
\renewcommand*{\bibfont}{\small}

% The following packages can be found on http:\\www.ctan.org
%\usepackage{graphics} % for pdf, bitmapped graphics files
%\usepackage{epsfig} % for postscript graphics files
%\usepackage{mathptmx} % assumes new font selection scheme installed
%\usepackage{mathptmx} % assumes new font selection scheme installed
%\usepackage{amsmath} % assumes amsmath package installed
%\usepackage{amssymb}  % assumes amsmath package installed

\title{\LARGE \bf
Game Research Lab: Abyssal Enigma Dive In Edition
}

\author{Tobias Brandner}

\affil{Julius-Maximilians University \\
        Würzburg, Germany \\
        tobias.brandner@stud-mail.uni-wuerzburg.de}

%\author{Huibert Kwakernaak$^{1}$ and Pradeep Misra$^{2}$% <-this % stops a space
%\thanks{*This work was not supported by any organization}% <-this % stops a space
%\thanks{$^{1}$H. Kwakernaak is with Faculty of Electrical Engineering, Mathematics and Computer Science,
    %    University of Twente, 7500 AE Enschede, The Netherlands
     %   {\tt\small h.kwakernaak at papercept.net}}%
%\thanks{$^{2}$P. Misra is with the Department of Electrical Engineering, Wright State University,
      %  Dayton, OH 45435, USA
       % {\tt\small p.misra at ieee.org}}%
%}


\begin{document}
\maketitle
\thispagestyle{empty}
\pagestyle{empty}

\BiblatexSplitbibDefernumbersWarningOff

%%%%%%%%%%%%%%%%%%%%%%%%%%%%%%%%%%%%%%%%%%%%%%%%%%%%%%%%%%%%%%%%%%%%%%%%%%%%%%%%
\begin{abstract}

    Horor VR stuff Abyssal Engima bla bla
    
\end{abstract}


%%%%%%%%%%%%%%%%%%%%%%%%%%%%%%%%%%%%%%%%%%%%%%%%%%%%%%%%%%%%%%%%%%%%%%%%%%%%%%%%
\section{Introduction}

%Subnautica reference why VR is awesome, some shit like that
Adventure games play an essential role in game development, 
as the genre has influenced many of today's games, such as \textit{Starfield} \cite{starfield},
and can be dated back decades, all the way to the infamous game \textit{The Secret of Monkey Island} \cite{monkey}.
Since adventure games combine several traits such as exploration, storytelling and puzzle solving, several sub-genres have developed over the years to differentiate and classify the numerous games.
One of these sub-genres is called \textbf{Walking Simulator}, which explicitly focuses on exploring the game world and its narrative \cite{adventure}.
\textit{Abyssal Enigma} is a first-person underwater exploration game with horror elements and can therefore best be described as a swimming simulator 
(as it literally offers depth in the form of free movement in a 3D environment).
%GRL split in 2 parts -> ISPN and VR port
This GRL is divided into two parts:
The first part was the creation of the game - \textit{Abyssal Enigma} - as part of the university course \textit{Interactive Stories and Playable Narratives (ISPN)}, 
a cooperation between the Julius-Maximilians-Universität (JMU) and the Technische Hochschule Würzburg (THW),
and the development by 6 students (including myself)\footnote{for more information see \ref{Sec:Acknowledgement}}. 
As the name of the course suggests, the game focuses heavily on narrative and can be summarized as follows:

\begin{quote}
    You play Charlie, an underwater researcher investigating the mysterious death of his mentor.
    Dive deeper to explore the scheming of the Subnautical Research Institute.
\end{quote} 

In the second part, I  reworked the game - \textit{Abyssal Enigma Dive In Edition} - to make it playable in virtual reality (VR).
Since it's not uncommon for exploration games to get a VR port at a later date, as \textit{Subnautica} \cite{Subnautica} proved, 
which incidentally served as the biggest inspiration for this project.
Both the desktop version and the VR port can be downloaded and played at
\url{https://miggli.itch.io/abyssal-enigma}.

In the following section \ref{Sec:Game} we focus on the development of the desktop version of the game, 
especially on the key components such as the player character or the O2 system.
In the \ref{Sec:VR} section, we go into the necessary steps to convert the desktop version into a functional VR version,
how to mitigate motion sickness 
and focus on the most important development challenges, such as the player character or the user interface (UI).
In the \ref{Sec:Performance} section, the VR version is briefly evaluated in terms of its performance.
Finally, in section \ref{Sec:ConcFuture} we conclude our work in the form of insights and potential future work.

\section{game development}
\label{Sec:Game}
% Unreal Engine 5.1
\textit{Abyssal Enigma} was developed with Unreal Engine 5.1\footnote{\url{https://www.unrealengine.com/en-US/blog/unreal-engine-5-1-is-now-available}},
as the setting of the game focuses on a realistic underwater world.
Therefore we explored a lot with different shaders simulating realistic underwater fog 
and used Quixel Mega Scan\footnote{\url{https://quixel.com/megascans}} assets for corals, stones and a sunken plane for example. 
% Player character
For the player character we used a space suit (since there are a lot more free spacesuits available than diver suits) 
and tweaked the color in Blender and reworked the boots to fins.
For realistic swim animations we used an asset pack and used Unreal's retargeting system to allows us to apply animations developed for
other characters to our character \cite{retarget}.
To complete the visuals we worked on a particle system representing bubbles and attached them to the hands of the character, further increasing the underwater immersion.
This summarizes to a true first person experience as seen in image \ref{fig:oldPlayer}, since there is an complete rigged model attached to the camera instead of just the arms.

\begin{figure}[!ht]
    \caption{True first person character with a fully rigged suit, bubble particles systems attached to the hand bones and the camera attached to the head bone.}
    \centering
    \includegraphics[width=0.6\textwidth]{images/oldPlayer.png}
    \label{fig:oldPlayer}
\end{figure}

% Levels
The game consists of 5 consecutive levels:
\begin{enumerate}
    \item The start screen showing the research ship on the water surface
    \item A coral riff right beneath the surface
    \item An algae forest in dirty churned-up water
    \item a deep sea volcanic rift inhabiting a gigantic monster
    \item a vast dessert
\end{enumerate}


% O2 system
% Assets and placement
% Integrating voice overs

\section{VR Port}
\label{Sec:VR}
% Unreal Engine 5.3
Unreal VR template
Reworking character
Motion Sickness
Anti motion sickness techniques
UI

\section{Performance}
\label{Sec:Performance}

Short Performance analysis.
BalkenDiagram showing framerate of different levels + 13 ms line above is bad!

\section{Conclusion \& Future Work}
\label{Sec:ConcFuture}

UI should be redesigned completetely
More Movement Options explored in 3D VR fyling/swimming
Performance increased with DLSS and optimizations

\section*{Acknowledgement}
\label{Sec:Acknowledgement}

Miklas Neely for creating a diver's mask asset and Andreas Braun for contionous support on VR development.
And Missie Legral for his course 3DUI which was a great samasurium for does and donts in VR.
% \addtolength{\textheight}{-12cm}   % This command serves to balance the column lengths
                                  % on the last page of the document manually. It shortens
                                  % the textheight of the last page by a suitable amount.
                                  % This command does not take effect until the next page
                                  % so it should come on the page before the last. Make
                                  % sure that you do not shorten the textheight too much.

%%%%%%%%%%%%%%%%%%%%%%%%%%%%%%%%%%%%%%%%%%%%%%%%%%%%%%%%%%%%%%%%%%%%%%%%%%%%%%%%



%%%%%%%%%%%%%%%%%%%%%%%%%%%%%%%%%%%%%%%%%%%%%%%%%%%%%%%%%%%%%%%%%%%%%%%%%%%%%%%%



%%%%%%%%%%%%%%%%%%%%%%%%%%%%%%%%%%%%%%%%%%%%%%%%%%%%%%%%%%%%%%%%%%%%%%%%%%%%%%%%


\printbibliography[
title={References},
notkeyword=game
]

\printbibliography[
title={Ludography},
keyword=game
]
%\printbibliography

% \begin{thebibliography}{99}

% \bibitem{c1} G. O. Young, ``Synthetic structure of industrial plastics (Book style with paper title and editor),'' 	in Plastics, 2nd ed. vol. 3, J. Peters, Ed.  New York: McGraw-Hill, 1964, pp. 15--64.
% \bibitem{c2} W.-K. Chen, Linear Networks and Systems (Book style).	Belmont, CA: Wadsworth, 1993, pp. 123--135.
% \bibitem{c3} H. Poor, An Introduction to Signal Detection and Estimation.   New York: Springer-Verlag, 1985, ch. 4.
% \bibitem{c4} B. Smith, ``An approach to graphs of linear forms (Unpublished work style),'' unpublished.
% \bibitem{c5} E. H. Miller, ``A note on reflector arrays (Periodical styleÑAccepted for publication),'' IEEE Trans. Antennas Propagat., to be publised.
% \bibitem{c6} J. Wang, ``Fundamentals of erbium-doped fiber amplifiers arrays (Periodical styleÑSubmitted for publication),'' IEEE J. Quantum Electron., submitted for publication.
% \bibitem{c7} C. J. Kaufman, Rocky Mountain Research Lab., Boulder, CO, private communication, May 1995.
% \bibitem{c8} Y. Yorozu, M. Hirano, K. Oka, and Y. Tagawa, ``Electron spectroscopy studies on magneto-optical media and plastic substrate interfaces(Translation Journals style),'' IEEE Transl. J. Magn.Jpn., vol. 2, Aug. 1987, pp. 740--741 [Dig. 9th Annu. Conf. Magnetics Japan, 1982, p. 301].
% \bibitem{c9} M. Young, The Techincal Writers Handbook.  Mill Valley, CA: University Science, 1989.
% \bibitem{c10} J. U. Duncombe, ``Infrared navigationÑPart I: An assessment of feasibility (Periodical style),'' IEEE Trans. Electron Devices, vol. ED-11, pp. 34--39, Jan. 1959.
% \bibitem{c11} S. Chen, B. Mulgrew, and P. M. Grant, ``A clustering technique for digital communications channel equalization using radial basis function networks,'' IEEE Trans. Neural Networks, vol. 4, pp. 570--578, July 1993.
% \bibitem{c12} R. W. Lucky, ``Automatic equalization for digital communication,'' Bell Syst. Tech. J., vol. 44, no. 4, pp. 547--588, Apr. 1965.
% \bibitem{c13} S. P. Bingulac, ``On the compatibility of adaptive controllers (Published Conference Proceedings style),'' in Proc. 4th Annu. Allerton Conf. Circuits and Systems Theory, New York, 1994, pp. 8--16.
% \bibitem{c14} G. R. Faulhaber, ``Design of service systems with priority reservation,'' in Conf. Rec. 1995 IEEE Int. Conf. Communications, pp. 3--8.
% \bibitem{c15} W. D. Doyle, ``Magnetization reversal in films with biaxial anisotropy,'' in 1987 Proc. INTERMAG Conf., pp. 2.2-1--2.2-6.
% \bibitem{c16} G. W. Juette and L. E. Zeffanella, ``Radio noise currents n short sections on bundle conductors (Presented Conference Paper style),'' presented at the IEEE Summer power Meeting, Dallas, TX, June 22--27, 1990, Paper 90 SM 690-0 PWRS.
% \bibitem{c17} J. G. Kreifeldt, ``An analysis of surface-detected EMG as an amplitude-modulated noise,'' presented at the 1989 Int. Conf. Medicine and Biological Engineering, Chicago, IL.
% \bibitem{c18} J. Williams, ``Narrow-band analyzer (Thesis or Dissertation style),'' Ph.D. dissertation, Dept. Elect. Eng., Harvard Univ., Cambridge, MA, 1993. 
% \bibitem{c19} N. Kawasaki, ``Parametric study of thermal and chemical nonequilibrium nozzle flow,'' M.S. thesis, Dept. Electron. Eng., Osaka Univ., Osaka, Japan, 1993.
% \bibitem{c20} J. P. Wilkinson, ``Nonlinear resonant circuit devices (Patent style),'' U.S. Patent 3 624 12, July 16, 1990. 

% \end{thebibliography}




\end{document}
