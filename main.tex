%%%%%%%%%%%%%%%%%%%%%%%%%%%%%%%%%%%%%%%%%%%%%%%%%%%%%%%%%%%%%%%%%%%%%%%%%%%%%%%%
%2345678901234567890123456789012345678901234567890123456789012345678901234567890
%        1         2         3         4         5         6         7         8

\documentclass[letterpaper, 10 pt, conference]{ieeeconf}  % Comment this line out
                                                          % if you need a4paper
%\documentclass[a4paper, 10pt, conference]{ieeeconf}      % Use this line for a4
                                                          % paper

\IEEEoverridecommandlockouts                              % This command is only
                                                          % needed if you want to
                                                          % use the \thanks command
\overrideIEEEmargins
% See the \addtolength command later in the file to balance the column lengths
% on the last page of the document

\usepackage[utf8]{inputenc}
\usepackage[T1]{fontenc}
% ref packages
%\usepackage{nameref}
%\usepackage{varioref}
% \usepackage{hyperref}
\usepackage[hyphens,spaces,obeyspaces]{url} % Fixes long URLs
\usepackage{hyperref} % Let URLs be clickable and lead to the website
\usepackage{hhline} % <-- Add this line to use hhline package
\usepackage{booktabs}
\usepackage{colortbl}
\usepackage{xcolor}
% \usepackage{enumitem} % Let enumerations start with the alphabet
\usepackage{graphicx}
\usepackage{floatrow}
\usepackage{float}
\usepackage{aliascnt}
\usepackage{multirow} % Tables
\usepackage{array}
\newcolumntype{P}[1]{>{\centering\arraybackslash}p{#1}}
\usepackage{authblk} % For the author section 
\usepackage{caption} % advanced captions
\usepackage{amsmath}
%Mathebib
\RequirePackage{amsmath}
\RequirePackage{amssymb}
\usepackage[
style=ieee, % Zitierstil
isbn=false,
doi=false,
url=false,
pagetracker=true,
%autocite=inline,  % regelt Aussehen für \autocite (inline=\parancite)
block=space,               
backref=false,
backrefstyle=three+,
date=year,
backend=biber
]{biblatex}
\addbibresource{literature.bib}
\renewcommand*{\bibfont}{\small}

% The following packages can be found on http:\\www.ctan.org
%\usepackage{graphics} % for pdf, bitmapped graphics files
%\usepackage{epsfig} % for postscript graphics files
%\usepackage{mathptmx} % assumes new font selection scheme installed
%\usepackage{mathptmx} % assumes new font selection scheme installed
%\usepackage{amsmath} % assumes amsmath package installed
%\usepackage{amssymb}  % assumes amsmath package installed

\title{\LARGE \bf
Game Research Lab: Abyssal Enigma Dive In Edition
}

\author{Tobias Brandner}

\affil{Julius-Maximilians University \\
        Würzburg, Germany \\
        tobias.brandner@stud-mail.uni-wuerzburg.de}

%\author{Huibert Kwakernaak$^{1}$ and Pradeep Misra$^{2}$% <-this % stops a space
%\thanks{*This work was not supported by any organization}% <-this % stops a space
%\thanks{$^{1}$H. Kwakernaak is with Faculty of Electrical Engineering, Mathematics and Computer Science,
    %    University of Twente, 7500 AE Enschede, The Netherlands
     %   {\tt\small h.kwakernaak at papercept.net}}%
%\thanks{$^{2}$P. Misra is with the Department of Electrical Engineering, Wright State University,
      %  Dayton, OH 45435, USA
       % {\tt\small p.misra at ieee.org}}%
%}


\begin{document}
\maketitle
\thispagestyle{empty}
\pagestyle{empty}

\BiblatexSplitbibDefernumbersWarningOff

%%%%%%%%%%%%%%%%%%%%%%%%%%%%%%%%%%%%%%%%%%%%%%%%%%%%%%%%%%%%%%%%%%%%%%%%%%%%%%%%
\begin{abstract}

    This Game Research Lab presents the development of "Abyssal Enigma Dive In Edition", a virtual reality (VR) adaptation of the original desktop game "Abyssal Enigma". 
    Using Unreal Engine 5.3, the VR version tackles the challenges associated with motion sickness through techniques such as head-controlled steering, a vignette effect, a virtual nose and the simulation of swimming movements.
    The desktop version focuses on narrative gameplay and offers a realistic underwater experience with guiding elements, an O2 mechanic and voice-over dialog. 
    In contrast, the VR port looks at the unique challenges of continuous movement in VR.
    The research concludes with insights into the potential of VR for adventure gaming, recognizing the importance of mitigating motion sickness. 
    Suggestions include reworking the user interface for VR, exploring realistic options for swimming motion, and optimizing performance through techniques such as aggressive culling and advanced upsampling methods such as DLSS.
    
\end{abstract}


%%%%%%%%%%%%%%%%%%%%%%%%%%%%%%%%%%%%%%%%%%%%%%%%%%%%%%%%%%%%%%%%%%%%%%%%%%%%%%%%
\section{Introduction}

%Subnautica reference why VR is awesome, some shit like that
Adventure games play an essential role in game development, 
as the genre has influenced many of today's games, such as \textit{Starfield} \cite{starfield},
and can be dated back decades, all the way to the infamous game \textit{The Secret of Monkey Island} \cite{monkey}.
Since adventure games combine several traits such as exploration, storytelling and puzzle solving, several sub-genres have developed over the years to differentiate and classify the numerous games.
One of these sub-genres is called \textbf{Walking Simulator}, which explicitly focuses on exploring the game world and its narrative \cite{adventure}.
\textit{Abyssal Enigma} is a first-person underwater exploration game with horror elements and can therefore best be described as a swimming simulator 
(as it literally offers depth in the form of free movement in a 3D environment).
%GRL split in 2 parts -> ISPN and VR port
This GRL is divided into two parts:
The first part was the creation of the game - \textit{Abyssal Enigma} - as part of the university course \textit{Interactive Stories and Playable Narratives (ISPN)}, 
a cooperation between the Julius-Maximilians-Universität (JMU) and the Technische Hochschule Würzburg (THW),
and the development by 6 students (including myself)\footnote{for more information see \ref{Sec:Acknowledgement}}. 
As the name of the course suggests, the game focuses heavily on narrative and can be summarized as follows:

\begin{quote}
    You play Charlie, an underwater researcher investigating the mysterious death of his mentor.
    Dive deeper to explore the scheming of the Subnautical Research Institute.
\end{quote} 

In the second part, I  reworked the game - \textit{Abyssal Enigma Dive In Edition} - to make it playable in virtual reality (VR).
Since it's not uncommon for exploration games to get a VR port at a later date, as \textit{Subnautica} \cite{Subnautica} proved, 
which incidentally served as the biggest inspiration for this project.
Both the desktop version and the VR port can be downloaded and played at
\url{https://miggli.itch.io/abyssal-enigma}.

In the following section \ref{Sec:Game} we focus on the development of the desktop version of the game, 
especially on the key components such as the player character or the O2 system.
In section \ref{Sec:VR}, we go into the necessary steps to convert the desktop version into a functional VR version,
how to mitigate motion sickness 
and focus on the most important development challenges, such as the player character or the user interface (UI).
In section \ref{Sec:Performance}, the VR version is briefly evaluated in terms of its performance.
Finally, in section \ref{Sec:ConcFuture} we conclude our work in the form of insights and potential future work.

\section{game development}
\label{Sec:Game}
% Unreal Engine 5.1
\textit{Abyssal Enigma} was developed with the Unreal Engine 5.1\footnote{\url{https://www.unrealengine.com/en-US/blog/unreal-engine-5-1-is-now-available}},
as the setting of the game focuses on a realistic underwater world.
Therefore, we experimented a lot with different shaders that simulate realistic underwater fog 
and used Quixel Mega Scan\footnote{\url{https://quixel.com/megascans}} assets for corals, rocks and a sunken airplane for example.

% Player character
We used a spacesuit for the player character (as there are many more free spacesuits than diving suits) 
and customized the color in Blender and turned the boots into fins.
For realistic swimming animations, we used an asset pack and used Unreal's retargeting system to apply animations developed for other characters to our character \cite{retarget}.
To complete the look, we worked on a particle system to represent bubbles and attached them to the character's hands to further enhance the underwater immersion.
This results in a true first-person perspective, as can be seen in image \ref{fig:oldPlayer}, as a complete rigged model is attached to the camera and not just the arms.

\begin{figure}[!ht]
    \caption{True first person character with a fully rigged suit, bubble particles systems attached to the hand bones and the camera attached to the head bone.}
    \centering
    \includegraphics[width=0.6\textwidth]{images/oldPlayer.png}
    \label{fig:oldPlayer}
\end{figure}

% Levels
The game consists of 5 consecutive levels:
\begin{enumerate}
    \item The start screen showing the research ship on the water surface
    \item A coral riff right beneath the surface
    \item An algae forest in dirty churned-up water
    \item a deep sea volcanic rift inhabiting a gigantic monster
    \item a vast dessert
\end{enumerate}

Since the start screen only allows you to start the game, levels 2,3 and 4 allow free 3D movement in the form of swimming, while the last level only allows movement on the desert floor.
Therefore, an additional walking animation plus logic to switch between swimming and walking mode had to be implemented, which also hides the bubble particle systems.

% O2 system
To guide the player and limit 3D movement, we have implemented an O2 mechanic.
The O2 mechanic is a simple timer, the player character starts with 100\% O2 and loses either 0.5\% or 0.1\% every second depending on whether they move or not.
This results in a total time span of 200 seconds or ~3 minutes to 1000 seconds or ~16 minutes before drowning and respawning.
At a movement speed of 6 meters per second, the player could cover a maximum distance of 1200 meters in the game world.
(Note: The Unreal Engine uses centimeters, but this can easily be converted to meters). 
The maximum movement speed as well as the O2 consumption for movement and non-movement were stored in a data asset.
This allowed us to do two things:
(a) display the amount of time spent underwater and the distance traveled by the player, which allowed us to design and balance each level
and b) create multiple instances of this data object to change the player's behavior to our liking.
With this thought process, we placed O2 refill stations in moderately accessible areas and gave the player some room for exploration.

% Guiding elements
As mentioned earlier, the game tries to achieve a realistic underwater feel by adding a dense fog that restricts the player's vision.
Therefore, we have added orange orientation lights along the path for the player to explore and a UI marker for the next point of interest.
As soon as a point of interest is reached, the UI marker appears at the next position in the list.

% Integrating voice overs
As the game focuses primarily on its narrative, we have included logbooks at points of interest where the story is presented in text form.
In the game there are two co-workers, Juan and Sarah, who talk to Charlie from the research ship on the surface.
These dialogs help unravel the mystery and are displayed in the UI and fully voiced by two external voice actors (see \ref{Sec:Acknowledgement}).

\section{VR Port}
\label{Sec:VR}
% Unreal Engine 5.3
The VR version \textit{Abyssal Enigma Dive In Edition} has been ported and developed in Unreal Engine 5.3\footnote{\url{https://www.unrealengine.com/en-US/blog/unreal-engine-5-3-is-now-available}}.
In addition to some improvements in terms of Temporal Super Resolution (TSR), some rendering problems on AMD graphics cards have also been fixed.
% VR Setup
For the development process, I used an Oculus Quest in combination with SteamVR\footnote{\url{https://store.steampowered.com/app/250820/SteamVR/}} to stream my desktop to the headset.
With this setup, the Unreal Engine allows you to test the current scene in the editor instead of compiling the project.
This setup and feature allows for faster development and is overall a nice workflow.

% Unreal VR template
After that was clarified, I started working with VR in Unreal.
I used the standard VR template\footnote{\url{https://docs.unrealengine.com/5.0/en-US/vr-template-in-unreal-engine/}}, which is already equipped with a working VR character.
The character supports head and finger tracking (and therefore input mapping).
Unfortunately, the only movement option within the template is teleportation.
(The player points with one hand to a place on the floor where he wants to move immediately).
This form of movement is useless for free movement in 3D or swimming.
% Reworking character & Motion Sickness
So our only real option is to allow the player to move continuously in VR.
But that comes with a big problem: \textbf{Motion Sickness}.

\subsection{Anti motion sickness techniques}

Moving in VR but not in real life leads to a false interpretation of our brain, as we expect to move but we do not.
As a result, our brain thinks we are sick or have poisoned ourselves and therefore reacts with nausea.
To alleviate this problem, we have developed four techniques to combat motion sickness.

\subsubsection{Head-directed steering}

Head-directed steering has proven to mitigate motion sickness \cite{jerald2017vr}.
It only allows movement in the direction of the head, so the head controls the movement of the player character.
The big advantage of this technique is that it is easy to implement as I only need the forward vector of the camera for the direction.
The player can move in the direction of the head while moving the stick of one of the two controllers forward.
Another advantage is that it is quite intuitive to move freely in 3D this way.

\subsubsection{Vignette effect}

Reducing the field of view during movement in VR reduces the intensity of perceived movement and thus mitigates motion sickness \cite{basting2017effectiveness}.
To achieve this vignette effect seamlessly, I wrote a custom shader that uses Unreal's material collections \cite{collections}.
Material Collections allow shader variables to be stored and referenced within a collection instance. 
The collection allowed me to reference the vignette strength variable within my player character and steadily increase the effect the faster my character moved.

\subsubsection {Rest Frame - The virtual nose}

Having an object constantly in line of sight, which confirms you are resting is called a rest frame.
For example adding a virtual nose for a VR Jump'n Run mitigated motion sickness \cite{wienrich2018virtual}.
Our virtual nose is a divering mask and can be seen in \ref{fig:nose}

 \begin{figure}[!ht]
    \caption{Our virtual nose: A diver's mask is always in your line of sight, confirming you are resting.}
    \centering
    \includegraphics[width=0.8\textwidth]{images/nose.png}
    \label{fig:nose}
\end{figure}

\subsubsection{Simulating swim motion}

Since we are swimming in VR, I played with a simple idea to simulate a swimming motion.
I placed a collision box in front of the camera, but an arm's length away, as shown in figure \ref{fig:swim}.
The idea behind this is: 
If one or both hands enter the collision box and leave it again after a short time, about 0.75 seconds, I assume that the player wants to swim.
To do this, I move the player in the direction of the head for half a second or, if both hands are used, for a whole second.

\begin{figure}[!ht]
    \caption{Our simple swim motion idea: A collision box is attached to the camera.
    While one or both hands enter the collision box (changing the material of the hand to green) and leaving in short timespan of 0.75 seconds, the player will move in head direction.}
    \centering
    \includegraphics[width=0.8\textwidth]{images/swim.png}
    \label{fig:swim}
\end{figure}

\subsection{The UI problem}

Adding UI to the viewport, as in the desktop version of this game, does not work with VR as the headset has two displays and the UI does not render correctly.
In order for our UI to render correctly, we had to move it into world space, as shown in Figure \ref{fig:UI}.

\begin{figure}[!ht]
    \caption{Our player's UI in worlspace placed 1.2 meters away from the camera. We added contrast to the UI by adding a black background to the fine green lines.}
    \centering
    \includegraphics[width=0.8\textwidth]{images/UI.png}
    \label{fig:UI}
\end{figure}

We had to add a black background to our user interface to increase the contrast as it was hard to read in VR \cite{dingler2018vr}.
Since we were trying to port the UI from the desktop version, we had to make a lot of changes.
For example, when triggering dialogs, an animation plays in the UI and the dialog is displayed at the bottom of the screen.
As we added a rest frame in the shape of our diving mask, this was sometimes unreadable as the mask is closer to the camera than the UI.
Finding the perfect distance and making all UI states readable at the same time was a delicate task.

\section{Performance}
\label{Sec:Performance}

Since a stuttering VR experience led to motion sickness for me, I did a quick performance analysis.
First of all, the game only runs on my PC and not natively on the Quest headset.
I am using a 7900 XTX for the GPU and a Ryzen 7900 for the CPU.
The Oculus Quest has a framerate of 72 and a resolution of 2064 x 2272 per eye, according to SteamOverlay. 

\begin{figure}[!ht]
    \caption{Performance analysis of the frametime spikes of each level in Abyssal Enigma Dive In Edition. 
    As the Oculus Quest has a framerate of 72, a frametime of 13 mili seconds or below is important for a stutter free VR experience.}
    \centering
    \includegraphics[width=1.05\textwidth]{images/descent.png}
    \label{fig:descent}
\end{figure}

I played all levels with the Steam overlay turned on and noted 1\% lows (peaks in the frame time curve) to evaluate the performance of our game.
An important value is the number 13 milliseconds, as frametimes below this number ensure the 72 frame rate of the quest headset. 
In the graph \ref{fig:descent} we can see the 1\% lows of each of the 5 levels.
Only the start screen and the Algea Forest fall above 13 ms.
In the start screen this is due to the very elaborate shader for the water surface, but as the start screen does not allow any movement, this is not such a big problem.
The algae forest poses a bigger problem as it can have a lot of geometry in the line of sight and of course we are moving.
To mitigate the performance issues, we enabled TSR with a screen resolution of 50\%.
We tried to go lower, but the upsampling technique has trouble upsampling the space text, making it too hard to read.

\section{Conclusion \& Future Work}
\label{Sec:ConcFuture}

VR has great potential for adventure games and especially for the walking simulator subgenre, 
as it leads to more unique exploration experiences and supports immersion in the narrative.
Nevertheless, the problem of motion sickness should not be underestimated and should be treated with caution.
Even seemingly small effects, such as the vignette effect, can have a big impact.
Porting the UI to world space was the obvious choice, but not the best.
The UI should be reworked to be more suitable for VR, e.g. the O2 bar could be placed on the player's VR wrist and resemble a watch.
More complex movement options could be explored to simulate a more realistic swimming motion.
In terms of performance, more aggressive culling could be an option, and replacing TSR with Deep Learning Super Sampling (DLSS) with frame generation by the official plugin could lead to more stable image quality as well as an increase in frame rate.

\section*{Acknowledgement}
\label{Sec:Acknowledgement}
For the desktop game \textit{Abyssal Enigma} I would like to thank Andreas Braun, Felix Heimbeck, Johanna-Maria Zipper, Miklas Neely and Marianne Pouysegur.
For the VR version \textit{Abyssal Enigma Dive In Edition} I would like to thank Miklas Neely again for creating a diving mask and Andreas Braun for his support with the VR development.
Last but not least, I would like to thank Jean-Luc Lugrin for his 3D User Iinterfaces course, which was a great help for the development in VR.

% \addtolength{\textheight}{-12cm}   % This command serves to balance the column lengths
                                  % on the last page of the document manually. It shortens
                                  % the textheight of the last page by a suitable amount.
                                  % This command does not take effect until the next page
                                  % so it should come on the page before the last. Make
                                  % sure that you do not shorten the textheight too much.

%%%%%%%%%%%%%%%%%%%%%%%%%%%%%%%%%%%%%%%%%%%%%%%%%%%%%%%%%%%%%%%%%%%%%%%%%%%%%%%%



%%%%%%%%%%%%%%%%%%%%%%%%%%%%%%%%%%%%%%%%%%%%%%%%%%%%%%%%%%%%%%%%%%%%%%%%%%%%%%%%



%%%%%%%%%%%%%%%%%%%%%%%%%%%%%%%%%%%%%%%%%%%%%%%%%%%%%%%%%%%%%%%%%%%%%%%%%%%%%%%%


\printbibliography[
title={References},
notkeyword=game
]

\printbibliography[
title={Ludography},
keyword=game
]
%\printbibliography

% \begin{thebibliography}{99}

% \bibitem{c1} G. O. Young, ``Synthetic structure of industrial plastics (Book style with paper title and editor),'' 	in Plastics, 2nd ed. vol. 3, J. Peters, Ed.  New York: McGraw-Hill, 1964, pp. 15--64.
% \bibitem{c2} W.-K. Chen, Linear Networks and Systems (Book style).	Belmont, CA: Wadsworth, 1993, pp. 123--135.
% \bibitem{c3} H. Poor, An Introduction to Signal Detection and Estimation.   New York: Springer-Verlag, 1985, ch. 4.
% \bibitem{c4} B. Smith, ``An approach to graphs of linear forms (Unpublished work style),'' unpublished.
% \bibitem{c5} E. H. Miller, ``A note on reflector arrays (Periodical styleÑAccepted for publication),'' IEEE Trans. Antennas Propagat., to be publised.
% \bibitem{c6} J. Wang, ``Fundamentals of erbium-doped fiber amplifiers arrays (Periodical styleÑSubmitted for publication),'' IEEE J. Quantum Electron., submitted for publication.
% \bibitem{c7} C. J. Kaufman, Rocky Mountain Research Lab., Boulder, CO, private communication, May 1995.
% \bibitem{c8} Y. Yorozu, M. Hirano, K. Oka, and Y. Tagawa, ``Electron spectroscopy studies on magneto-optical media and plastic substrate interfaces(Translation Journals style),'' IEEE Transl. J. Magn.Jpn., vol. 2, Aug. 1987, pp. 740--741 [Dig. 9th Annu. Conf. Magnetics Japan, 1982, p. 301].
% \bibitem{c9} M. Young, The Techincal Writers Handbook.  Mill Valley, CA: University Science, 1989.
% \bibitem{c10} J. U. Duncombe, ``Infrared navigationÑPart I: An assessment of feasibility (Periodical style),'' IEEE Trans. Electron Devices, vol. ED-11, pp. 34--39, Jan. 1959.
% \bibitem{c11} S. Chen, B. Mulgrew, and P. M. Grant, ``A clustering technique for digital communications channel equalization using radial basis function networks,'' IEEE Trans. Neural Networks, vol. 4, pp. 570--578, July 1993.
% \bibitem{c12} R. W. Lucky, ``Automatic equalization for digital communication,'' Bell Syst. Tech. J., vol. 44, no. 4, pp. 547--588, Apr. 1965.
% \bibitem{c13} S. P. Bingulac, ``On the compatibility of adaptive controllers (Published Conference Proceedings style),'' in Proc. 4th Annu. Allerton Conf. Circuits and Systems Theory, New York, 1994, pp. 8--16.
% \bibitem{c14} G. R. Faulhaber, ``Design of service systems with priority reservation,'' in Conf. Rec. 1995 IEEE Int. Conf. Communications, pp. 3--8.
% \bibitem{c15} W. D. Doyle, ``Magnetization reversal in films with biaxial anisotropy,'' in 1987 Proc. INTERMAG Conf., pp. 2.2-1--2.2-6.
% \bibitem{c16} G. W. Juette and L. E. Zeffanella, ``Radio noise currents n short sections on bundle conductors (Presented Conference Paper style),'' presented at the IEEE Summer power Meeting, Dallas, TX, June 22--27, 1990, Paper 90 SM 690-0 PWRS.
% \bibitem{c17} J. G. Kreifeldt, ``An analysis of surface-detected EMG as an amplitude-modulated noise,'' presented at the 1989 Int. Conf. Medicine and Biological Engineering, Chicago, IL.
% \bibitem{c18} J. Williams, ``Narrow-band analyzer (Thesis or Dissertation style),'' Ph.D. dissertation, Dept. Elect. Eng., Harvard Univ., Cambridge, MA, 1993. 
% \bibitem{c19} N. Kawasaki, ``Parametric study of thermal and chemical nonequilibrium nozzle flow,'' M.S. thesis, Dept. Electron. Eng., Osaka Univ., Osaka, Japan, 1993.
% \bibitem{c20} J. P. Wilkinson, ``Nonlinear resonant circuit devices (Patent style),'' U.S. Patent 3 624 12, July 16, 1990. 

% \end{thebibliography}




\end{document}
