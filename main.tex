%%%%%%%%%%%%%%%%%%%%%%%%%%%%%%%%%%%%%%%%%%%%%%%%%%%%%%%%%%%%%%%%%%%%%%%%%%%%%%%%
%2345678901234567890123456789012345678901234567890123456789012345678901234567890
%        1         2         3         4         5         6         7         8

\documentclass[letterpaper, 10 pt, conference]{ieeeconf}  % Comment this line out
                                                          % if you need a4paper
%\documentclass[a4paper, 10pt, conference]{ieeeconf}      % Use this line for a4
                                                          % paper

\IEEEoverridecommandlockouts                              % This command is only
                                                          % needed if you want to
                                                          % use the \thanks command
\overrideIEEEmargins
% See the \addtolength command later in the file to balance the column lengths
% on the last page of the document

\usepackage[utf8]{inputenc}
\usepackage[T1]{fontenc}
% ref packages
%\usepackage{nameref}
%\usepackage{varioref}
% \usepackage{hyperref}
\usepackage[hyphens,spaces,obeyspaces]{url} % Fixes long URLs
\usepackage{hyperref} % Let URLs be clickable and lead to the website
\usepackage{hhline} % <-- Add this line to use hhline package
\usepackage{booktabs}
\usepackage{colortbl}
\usepackage{xcolor}
% \usepackage{enumitem} % Let enumerations start with the alphabet
\usepackage{graphicx}
\usepackage{floatrow}
\usepackage{float}
\usepackage{aliascnt}
\usepackage{multirow} % Tables
\usepackage{array}
\newcolumntype{P}[1]{>{\centering\arraybackslash}p{#1}}
\usepackage{authblk} % For the author section 
\usepackage{caption} % advanced captions
%Mathebib
\RequirePackage{amsmath}
\RequirePackage{amssymb}
\usepackage[
style=ieee, % Zitierstil
isbn=false,
doi=false,
url=false,
pagetracker=true,
%autocite=inline,  % regelt Aussehen für \autocite (inline=\parancite)
block=space,               
backref=false,
backrefstyle=three+,
date=year,
backend=biber
]{biblatex}
\addbibresource{literature.bib}
\renewcommand*{\bibfont}{\small}

% The following packages can be found on http:\\www.ctan.org
%\usepackage{graphics} % for pdf, bitmapped graphics files
%\usepackage{epsfig} % for postscript graphics files
%\usepackage{mathptmx} % assumes new font selection scheme installed
%\usepackage{mathptmx} % assumes new font selection scheme installed
%\usepackage{amsmath} % assumes amsmath package installed
%\usepackage{amssymb}  % assumes amsmath package installed

\title{\LARGE \bf
Game Research Lab: Investigating Movement Flow in 3D Jump'n Runs
}

\author{Tobias Brandner}

\affil{Julius-Maximilians University \\
        Würzburg, Germany \\
        tobias.brandner@stud-mail.uni-wuerzburg.de}

%\author{Huibert Kwakernaak$^{1}$ and Pradeep Misra$^{2}$% <-this % stops a space
%\thanks{*This work was not supported by any organization}% <-this % stops a space
%\thanks{$^{1}$H. Kwakernaak is with Faculty of Electrical Engineering, Mathematics and Computer Science,
    %    University of Twente, 7500 AE Enschede, The Netherlands
     %   {\tt\small h.kwakernaak at papercept.net}}%
%\thanks{$^{2}$P. Misra is with the Department of Electrical Engineering, Wright State University,
      %  Dayton, OH 45435, USA
       % {\tt\small p.misra at ieee.org}}%
%}


\begin{document}
\maketitle
\thispagestyle{empty}
\pagestyle{empty}

\BiblatexSplitbibDefernumbersWarningOff

%%%%%%%%%%%%%%%%%%%%%%%%%%%%%%%%%%%%%%%%%%%%%%%%%%%%%%%%%%%%%%%%%%%%%%%%%%%%%%%%
\begin{abstract}

blub

\end{abstract}


%%%%%%%%%%%%%%%%%%%%%%%%%%%%%%%%%%%%%%%%%%%%%%%%%%%%%%%%%%%%%%%%%%%%%%%%%%%%%%%%
\section{Introduction}

Super Mario 64 \cite{SuperMario64} is one of the first 3D Jump'n Runs and was released in 1995.
To this day it is one of the most played games in terms of speedrunning \cite{SpeedrunDotCom}.
A speedrun is a player's attempt to finish a game as quickly as possible.
To accomplish a successful speedrun, a player needs deep knowledge of the game, its mechanics, and its controls \cite{speedrun}.
But what drives people to play this or any other game over and over again? 
Looking at a successful speedrun of Super Mario 64 120 stars \cite{Sm64Speed} done at Games Done Quick\footnote{\url{https://gamesdonequick.com/}}, a charity video game speedrun marathon, one can observe an excellent level of gameplay and flow. 
\citeauthor{swink2008game} defines game flow or game feel in his book \textit{Game Feel} as:

\begin{quote}
    Real-time control of virtual objects in a simulated space, emphasizing interactions through polish \cite{swink2008game}.
\end{quote}

Game feel is built on three blocks:

\begin{enumerate}
    \item Real-Time Control, e.g., the input processing of the player entity or entities.
    \item Spatial Simulation, e.g., the level design, especially the arrangement of collisions and response systems.
    \item Polish, e.g., sound effects, animations, and particle effects played when interacting with the world
\end{enumerate}

With respect to 3D Jump'n Runs, this means reasonable control and response of our player character, a game world that matches the player's movement parameters, e.g., his jump height and distance, and audio-visual cues that emphasize the player character's interactions in the world.
To investigate a good movement flow, we first developed a set of movement mechanics, which are explained in Section \ref{Sec:Met}.
We also investigate various presets for the movement parameters on which the mechanics are built.
To do so, we visit different games and compare them in Section \ref{Sec:Evaluation}.
We conclude our findings in section \ref{Sec:ConcFuture} and outline some future work.

\section{Methodology}
\label{Sec:Met}

Before we implement any movement mechanics in \textit{Unreal Engine 5}\footnote{\url{https://www.unrealengine.com/en-US}}, we first need to talk about all the requirements necessary to create a dynamic player character.
Then we talk about the polishing steps to make the character more connected to the game world.

\subsection{Prerequisites}

To create a dynamic player character, we used \textit{Blender}\footnote{\url{https://www.blender.org/}}, a 3D modeling software, to model and design our character.
After this process, we obtained a model with a high density of vertices, which is unsuitable for use in a real-time environment because it is too computationally intensive.
Therefore, we need to retopologize the model to create a more uniform and less dense version (up to 100,000 vertices is fine).
We rigged the character using \textit{accuRig}\footnote{\url{https://actorcore.reallusion.com/auto-rig}}, an automatic rigging tool.
Rigging describes the process of creating a skeleton for a 3D model so that it can be more easily animated \cite{rigging}.
Instead of creating our own animations for our character's various movement options, which would be incredibly time consuming, we used Unreal's retargeting mechanic \cite{retarget}.
IK Retargeting allows us to apply animations developed for other characters to our character.
With this, we now had all the tools we needed to create a compelling player character.

\subsection{Movement mechanics}

An overview of all the movement mechanics we have implemented in Unreal Engine 5 can be seen in figure \ref{fig:mechanics}.

\begin{figure}[!ht]
    \caption{Hierachically structured taxonomy of all movement states that the player character can enter in the game world.
    }
    \centering
    \includegraphics[width=0.8\textwidth]{images/mechanics.png}
    \label{fig:mechanics}
\end{figure}

The most basic function is the idle or walk state, which the character enters while moving.
From there, the player can perform the triple jump mechanic.
When the player presses the jump button, the character initially enters the normal jump state. 
After the character lands, the player has a window of $\textit{0.25s}$ to press the jump button again.
This allows the character to transition to the double jump state and then to the triple jump state.
These three jump states differ only in the speed at which the character is flung into the air. \\
If the character is in the idle/walking state and holds down the sprint button, he can run faster and switch to the sprint state.
Now when you perform a jump, it becomes a long jump.
The long jump catapults the character into the air with the least amount of force, but because of the faster ground movement, the player can cover greater distances with this type of jump. \\
When the character jumps against a wall, he enters the wall slide state. 
This mechanic slows the character's gravity drag in the air and near a wall.
From there, the player can press the climb button to enter the climbing state.
In the climbing state, gravity does not affect the character, and he can move freely on the surface of a climbable game object.
In each frame, the climbing logic performs a collision check to see if a climbable surface is still within reach of the character.
The player can move the character away from a valid query, resulting in an automatic exit from the climbing state, i.e., the player climbs up a wall and at the end the character automatically performs a climb-up animation and exits the climbing state.
While sliding or climbing the wall, the player can also press the jump button, which triggers the wall jump.
When wall jumping, the character is launched not only with a velocity upwards, but also with a velocity to the side, away from the wall.
By default, the character is launched in the direction of the normal vector of the queried surface.
The mechanics of the wall jump gives the player further control by allowing them to control the launch angle with respect to the wall, i.e. they can even make the character jump along the wall.
This concludes the rough overview of all the movement mechanics we have implemented.

\subsection{Polish}

To highlight the different states and their transitions, we added animations and blendspaces to all of them.
For example, the transition from idle to walking and sprinting is done with a blendspace that automatically transitions between the three different animations based on a value, in this case the character's movement speed.
An animation graph (in Unreal it is called Animation Blueprint) takes care of the transition between the different animations/blendspaces based on transition rules, such as whether the character is in the air.
These game logic states are injected by the character's blueprint.
To make the character's interaction with the world more immersive, we added particle effects for the different states.
For his steps when he lands on the ground and while he slides down a wall, we added a dust cloud particle effect.
While he climbs, we added a pebble particle effect.
All particle effects get stronger the more force is applied to the character, so when he slides down a wall, the dust cloud gets bigger the faster the character slides down.
Last but not least, we have added different sound effects for the different jumps or steps/climbing.


\section{Evaluation}
\label{Sec:Evaluation}

To evaluate the flow of movement for the mechanics implemented so far, we need to look at the parameters that lie at the base of these mechanics.
For example, the walking mechanic of a player character is defined by \textit{a)} its acceleration \textit{b)} its maximum speed that can be reached, \textit{c)} the deceleration to come back to a stop.
To get a good feel for the different parameters, we studied other famous games that rely heavily on their movement mechanics and collected their data.
Then we compared these different motion parameter presets by plotting them against each other.
Finally, we evaluate the motion flow based on the feedback from the players that we collected in a play session.

\subsection{Gathering data from other games}

\subsection{Comparing movement parameters}

% \begin{table*}[htbp]
%     \centering
%     \begin{tabular}{cccccc}
%     \toprule
%      & \textbf{Boss'n Run} & \textbf{Super Mario 64} & \textbf{Donkey Kong 64} & \textbf{Banjo Kazooie} \\
%     \midrule
%     \textbf{Velocity v ($\frac{\mathrm{m}}{\mathrm{s}}$)} & \textcolor{red}{6} & \textcolor{blue}{4} & 5.4 & 4.5 \\
%     \textbf{Acceleration acc ($\frac{\mathrm{m}}{\mathrm{s^2}}$)} & \textcolor{red}{20} & \textcolor{blue}{8} & 12.8 & 18 \\
%     \textbf{Time t ($\mathrm{s}$)} & 0.3 & \textcolor{blue}{0.5} & 0.42 & \textcolor{red}{0.25} \\
%     \textbf{Distance d ($\mathrm{m}$)} & 0.9 & 1.0 & \textcolor{blue}{1.14} & \textcolor{red}{0.56} \\
%     \bottomrule
%     \end{tabular}
%     \caption{A 4x4 Table with Improved Formatting and Corrected Color Coding}
%     \label{tab:insightI}
% \end{table*}


\begin{table*}[htbp]
    \centering
    \begin{tabular}{cccccc}
    \toprule
     & \textbf{Boss'n Run} & \textbf{Super Mario 64} & \textbf{Donkey Kong 64} & \textbf{Banjo Kazooie} \\
    \midrule
    \textbf{Velocity v ($\frac{\mathrm{m}}{\mathrm{s}}$)} & \textcolor{red}{6} & \textcolor{blue}{4} & 5.4 & 4.5 \\
    \textbf{Acceleration acc ($\frac{\mathrm{m}}{\mathrm{s^2}}$)} & \textcolor{red}{20} & \textcolor{blue}{8} & 12.8 & 18 \\
    \textbf{Deceleration dec ($\frac{\mathrm{m}}{\mathrm{s^2}}$)} & \textcolor{red}{20} & \textcolor{blue}{8} & 12.8 & 18 \\
    \textbf{Time t ($\mathrm{s}$)} & 1.1 & \textcolor{blue}{1.5} & 1.35 & \textcolor{red}{1.0} \\
    \textbf{Distance d ($\mathrm{m}$)} & 4.8 & 4 & \textcolor{blue}{4.98} & \textcolor{red}{3.38} \\
    \bottomrule
    \end{tabular}
    \caption{Movement parameters of the 4 games Boss'n Run, Super Mario 64, Donkey Kong 64 and Banjo Kazooie. 
    For speed, acceleration and deceleration, the highest values are marked in red and the lowest values in blue.
    For time and distance to reach maximum speed, the lowest values are marked in red and the highest values in blue.}
    \label{tab:insightII}
\end{table*}


\begin{table*}[htbp]
    \centering
    \begin{tabular}{cccccc}
    \toprule
     & \textbf{Boss'n Run} & \textbf{Super Mario 64} & \textbf{Donkey Kong 64} & \textbf{Banjo Kazooie} \\
    \midrule
    \textbf{Gravity g ($\frac{\mathrm{m}}{\mathrm{s^2}}$)} & -18.6 & \textcolor{red}{-20.7} & -20.2 & \textcolor{blue}{-14.0} \\
    \textbf{Ground Velocity $v_{\text{g}}$ ($\frac{\mathrm{m}}{\mathrm{s}}$)} & \textcolor{red}{6} & \textcolor{blue}{4} & 5.4 & 4.5 \\
    \textbf{Jump Velocity $v_{\text{j}}$ ($\frac{\mathrm{m}}{\mathrm{s}}$)} & 8.7 & \textcolor{red}{9.3} & 8.2 & \textcolor{blue}{7.3} \\
    \textbf{Height h ($\mathrm{m}$)} & 2.03 & \textcolor{red}{2.09} & \textcolor{blue}{1.66} & 1.84 \\
    \textbf{Time t ($\mathrm{s}$)} & 0.93 & 0.9 & \textcolor{blue}{0.81} & \textcolor{red}{1.0} \\
    \textbf{Distance d ($\mathrm{m}$)} & \textcolor{red}{5.6} & \textcolor{blue}{3.6} & 4.4 & 4.5 \\
    \bottomrule
    \end{tabular}
    \caption{Default jump parameters of the 4 games Boss'n Run, Super Mario 64, Donkey Kong 64 and Banjo Kazooie.
    For all parameters, the highest values are marked in red and the lowest in blue.}
    \label{tab:insightIII}
\end{table*}


\subsection{Feedback from players}


\section{Conclusion \& Future Work}
\label{Sec:ConcFuture}

blub

% \addtolength{\textheight}{-12cm}   % This command serves to balance the column lengths
                                  % on the last page of the document manually. It shortens
                                  % the textheight of the last page by a suitable amount.
                                  % This command does not take effect until the next page
                                  % so it should come on the page before the last. Make
                                  % sure that you do not shorten the textheight too much.

%%%%%%%%%%%%%%%%%%%%%%%%%%%%%%%%%%%%%%%%%%%%%%%%%%%%%%%%%%%%%%%%%%%%%%%%%%%%%%%%



%%%%%%%%%%%%%%%%%%%%%%%%%%%%%%%%%%%%%%%%%%%%%%%%%%%%%%%%%%%%%%%%%%%%%%%%%%%%%%%%



%%%%%%%%%%%%%%%%%%%%%%%%%%%%%%%%%%%%%%%%%%%%%%%%%%%%%%%%%%%%%%%%%%%%%%%%%%%%%%%%


\printbibliography[
title={References},
notkeyword=game
]

\printbibliography[
title={Ludography},
keyword=game
]
%\printbibliography

% \begin{thebibliography}{99}

% \bibitem{c1} G. O. Young, ``Synthetic structure of industrial plastics (Book style with paper title and editor),'' 	in Plastics, 2nd ed. vol. 3, J. Peters, Ed.  New York: McGraw-Hill, 1964, pp. 15--64.
% \bibitem{c2} W.-K. Chen, Linear Networks and Systems (Book style).	Belmont, CA: Wadsworth, 1993, pp. 123--135.
% \bibitem{c3} H. Poor, An Introduction to Signal Detection and Estimation.   New York: Springer-Verlag, 1985, ch. 4.
% \bibitem{c4} B. Smith, ``An approach to graphs of linear forms (Unpublished work style),'' unpublished.
% \bibitem{c5} E. H. Miller, ``A note on reflector arrays (Periodical styleÑAccepted for publication),'' IEEE Trans. Antennas Propagat., to be publised.
% \bibitem{c6} J. Wang, ``Fundamentals of erbium-doped fiber amplifiers arrays (Periodical styleÑSubmitted for publication),'' IEEE J. Quantum Electron., submitted for publication.
% \bibitem{c7} C. J. Kaufman, Rocky Mountain Research Lab., Boulder, CO, private communication, May 1995.
% \bibitem{c8} Y. Yorozu, M. Hirano, K. Oka, and Y. Tagawa, ``Electron spectroscopy studies on magneto-optical media and plastic substrate interfaces(Translation Journals style),'' IEEE Transl. J. Magn.Jpn., vol. 2, Aug. 1987, pp. 740--741 [Dig. 9th Annu. Conf. Magnetics Japan, 1982, p. 301].
% \bibitem{c9} M. Young, The Techincal Writers Handbook.  Mill Valley, CA: University Science, 1989.
% \bibitem{c10} J. U. Duncombe, ``Infrared navigationÑPart I: An assessment of feasibility (Periodical style),'' IEEE Trans. Electron Devices, vol. ED-11, pp. 34--39, Jan. 1959.
% \bibitem{c11} S. Chen, B. Mulgrew, and P. M. Grant, ``A clustering technique for digital communications channel equalization using radial basis function networks,'' IEEE Trans. Neural Networks, vol. 4, pp. 570--578, July 1993.
% \bibitem{c12} R. W. Lucky, ``Automatic equalization for digital communication,'' Bell Syst. Tech. J., vol. 44, no. 4, pp. 547--588, Apr. 1965.
% \bibitem{c13} S. P. Bingulac, ``On the compatibility of adaptive controllers (Published Conference Proceedings style),'' in Proc. 4th Annu. Allerton Conf. Circuits and Systems Theory, New York, 1994, pp. 8--16.
% \bibitem{c14} G. R. Faulhaber, ``Design of service systems with priority reservation,'' in Conf. Rec. 1995 IEEE Int. Conf. Communications, pp. 3--8.
% \bibitem{c15} W. D. Doyle, ``Magnetization reversal in films with biaxial anisotropy,'' in 1987 Proc. INTERMAG Conf., pp. 2.2-1--2.2-6.
% \bibitem{c16} G. W. Juette and L. E. Zeffanella, ``Radio noise currents n short sections on bundle conductors (Presented Conference Paper style),'' presented at the IEEE Summer power Meeting, Dallas, TX, June 22--27, 1990, Paper 90 SM 690-0 PWRS.
% \bibitem{c17} J. G. Kreifeldt, ``An analysis of surface-detected EMG as an amplitude-modulated noise,'' presented at the 1989 Int. Conf. Medicine and Biological Engineering, Chicago, IL.
% \bibitem{c18} J. Williams, ``Narrow-band analyzer (Thesis or Dissertation style),'' Ph.D. dissertation, Dept. Elect. Eng., Harvard Univ., Cambridge, MA, 1993. 
% \bibitem{c19} N. Kawasaki, ``Parametric study of thermal and chemical nonequilibrium nozzle flow,'' M.S. thesis, Dept. Electron. Eng., Osaka Univ., Osaka, Japan, 1993.
% \bibitem{c20} J. P. Wilkinson, ``Nonlinear resonant circuit devices (Patent style),'' U.S. Patent 3 624 12, July 16, 1990. 

% \end{thebibliography}




\end{document}
